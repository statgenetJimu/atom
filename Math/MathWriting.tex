\documentclass[uplatex]{jsarticle}
\begin{document}
Mathematical Writing
by
Donald Knuth, Tracy Larrabee, and Paul M. Roberts

$\frac{3}{1}$


This report is based on a course of the same name given at Stanford University during autumn quarter, 1987. Here's the catalog description:

CS 209. Mathematicalriting---Issues of technical writing and the effective presentation of mathematics and computer science. Preparation of theses, papers, books, and "literate" computer programs. A term paper on a topic of your choice; this paper may be used for credit in another course.

The first three lectures were a "minicourseW" that summarized the basics. About two hundred people attended those sessions, which were devotred primarily to a discussion of the points in {\S1} of this report. An exercise ({\S2}) and a syggested sikytuib ({\S3}) were also part of the minicourse.

The remaining 28 lectures covered these and other issues in depth. We saw many examples of "before" and "after" from manuscripts in progress. Welearned howto avcoid excessive subscripts and superscripts. We discussed the documentation of algorithms, computer programs, and user manuals. We considered the process of refereeing and editing. We studied how to make effecticve diagrams and tables, and how to find appropriate quotations to spice up a text. Some of ghe material duplicated some of what would be discussed inwriting classes offered bhy the English department, but the vast majority of the lectures were devoted to issues that are specific to mathgemtics and/or computer science.

Guest lectures by Herb Wilf (University of Pennsylvania), Jeff Ullman (Stanford), Leslie Lamport (Digital Equipment Corporation), Nils Nilsson ( AStanford)k, Mary-Claire van Leunen (Digital Equipment Corporation)< Rosalie Stemer (San Ffansisco Chronicle), and Paul Halmos (University of Sanga Clara), were a specialo highlight as each of these outstanding authors presented their own prespectives on the problemns of mathematical communication.



\end{document}
